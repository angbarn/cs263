\section{Proposed system}

\subsection{Security measures}
\subsubsection{Customer support verification}

One common attack vector for bank customers is impersonating a member ofcustomer
support. Similarly, a member of customer support could potentially ``go rogue'' 
and attack a customer’s account.

While many consider phonelines to be a safe method of communication, ``vishing''
or ``voice phishing'' is on the rise. \cite{bbcPhone} Phone number spoofing can make an
illegitimate call appear to be from a legitimate source. Holding open a line can
redirect a call from a customer to their bank towards an attacker.

Such techniques mean that verification of both Wondough Bank’s staff members and customers is necessary in every situation. The proposed system to satisfy such a goal is displayed in Figure 2.

First, a customer contacts a member of customer support through a potentially insecure connection, such as a phone call. Alternatively, a member of customer support may contact the customer first. (step 1)

To authenticate to the customer, the staff member’s customer service terminal will provide a one-time access code (OTAC) generated from the time, the staff member’s credentials, and the customer’s credentials. The staff member passes this to a customer, who verifies it using the Wondough Bank phone app. (step 2-5)

After verifying the customer support member’s OTAC, the phone app will provide a customer’s own OTAC. They must then provide this to the customer support representative. (step 6-8)

The customer service web application will send both OTACs to the customer support server, which will verify the codes server-side, before granting the staff member access to the customer’s account.

Such a system requires authorisation and consent from both the staff member and the customer before any changes to a customer’s account may be made.
